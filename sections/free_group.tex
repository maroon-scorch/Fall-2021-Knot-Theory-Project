\section{Free Group and Group Presentations}
Much of this section was based on Paolo Aluffi's \textit{Algebra : Chapter 0} \cite{aluffi2021algebra}.\\\\
The idea of free groups arose from the question: given an arbitrary set A, what is the most ``efficient" way to construct a group, which we will call $F(A)$, from the set in the sense that there's as little amount of relations as possible (hence the name ``Free") in the constructed group based on the set A.\\\\
One can imagine that in this new F(A), each element would essentially correspond to a random combination of elements from the set A itself. Indeed, that's where the motivation for a free group arises from.
\begin{definition}[Letters] Let A be a set, and let $$A^{-1} = \{a^{-1} : a \in A\}$$ be the set of ``inverse symbols" of A, then we call the combined set $A \bigcup A^{-1}$ the letters of A.\\
\end{definition}
\begin{definition}[Word] A word w from A is a finitely written product of elements from $A\bigcup A^{-1}$, ie, a word of length n can be expressed as:
\[w = a_0a_1...a_n\]
for $a_i = b_i^{\pm 1}$ and $b_i \in A$\\\\
For example, if $A = \{a, b\}$, then $w = abaab^{-1}$ is a word of length 5.\\\\
The product of zero letters is also a word; it's called the empty word.\\\\
Note that since there's no group properties yet, the word $w_1 = bb^{-1}$ and the word $w_2 = aa^{-1}$ are two different words.\\
\end{definition}

\begin{definition}[Reduction] Given a word w from A, we define a reduction operation $r: W(A) \to W(A)$ to be the process where r finds the first occurrence of $a^{-1}a$ or $aa^{-1}$ for any $a \in A$ in w and removes that term from w. If the word w has no occurrences of $a^{-1}a$ or $aa^{-1}$ for any $a \in A$, then $r(w) = w$.\\\\
For example, if $A = \{a, b, c\}$, for the word $w = abcc^{-1}baa^{-1}$, then $r(w) = abbaa^{-1}$, and $r(r(w)) = abb$.\\
\end{definition}

\begin{definition}[Reduced Word]
Let w be a word of length n from the set A. Then we call $r^{\lfloor \frac{n}{2} \rfloor}(w)$ a reduced word.
\end{definition}
\noindent The main idea of a reduced word is to get rid of the only relation in a group that we know for sure of that when the letter and its inverse are adjacent to one another, they should ideally become the identity in our group construction. A reduced word is essentially a word where no said adjacency can occur.\\\\
Consequently, one can see that the definition given for the reduced word has no said adjacency because in any word of length n, there are at most $\lfloor \frac{n}{2} \rfloor$ occurrences of this adjacency, which would all be removed by r.\\

\begin{definition}[Free Group]
Let $F(A)$ be the set of all reduced words of A. Then one can verify that $F(A)$ is a group under the operation {\bf concatenation-reduction} where, for any $w, v \in F(A)$, $w = a_0a_1.._a_n$, $v = b_0b_1...b_n$:
\[w \cdot_{F(A)} v = r(wv)\]
where the word $wv = a_0a_1..a_nb_0b_1...b_n$.\\\\
We call $F(A)$ with the operation given the {\bf free group} of A.\\
\end{definition}

\begin{example}
There are some common examples of free groups you may be familiar with:
\begin{enumerate}
    \item $F(\O)$ is the trivial group.
    \item $F(\{a\})$ consists of $\{..., a^{-1}a^{-1}, a^{-1}, \{\}, a, aa, ...\}$ is isomorphic to $\mathbb{Z}$.
    \item If A is a finite set of cardinality n, then we call $F(A)$ the free group of rank n.
\end{enumerate}
\end{example}

\noindent It turns out that every group G can be expressed as a quotient of a free group (just consider $F(G)$ and first isomorphism theorem), but often times one can express a group as a quotient of a free group generated with elements less than G. So in this senese, we can express every group G in terms of a sometimes simpler presentation of G in relations to a free group. This is also motivates the next definition below.
\begin{definition}[Group Presentation] Let A be some set and $R \subset F(A)$ with normal closure N in $F(A)$, then the natural homomorphism $\phi: F(A) \to G$ that evaluates each word as a multiplication of its letters in G, is clearly surjective and has kernel $ker(\phi) = N$.\\\\
Then we write $G = \frac{F(A)}{ker(\phi)}$ and denote its presentation as $G = \langle A | R \rangle = \langle A | r_1 = r_2 = ... = r_n = e \rangle$ for $r_i \in R$.\\\\
A is called the {\bf generators} of G, and R is called the ${\bf relations}$ of G.
\end{definition}
\noindent Since every group is the quotient of a free group, every group has a group presentation!
\begin{example} There are some familiar examples of presentations whose properties you may have seen already.
\begin{enumerate}
    \item Let $A = \{a_1, ..., a_n\}$, then $F(A) = \langle a_1, ..., a_n |\ \rangle$, that is, a free group has no relations.
    \item Let $n \in \mathbb{N}$, then $\mathbb{Z}/n\mathbb{Z} \cong \langle x\ |\ x^n = e\rangle$
    \item Let $D_{2n}$ be the dihedral group of the regular $n-gon$, then
    \[D_{2n} \cong \langle r,f\ |\ r^n = e, f^2 = e, frf = r^{-1}\rangle\]
    \item $\mathbb{Z} \times \mathbb{Z} \cong \langle a,b\ |\ ab = ba\rangle$
\end{enumerate}
\end{example}

\noindent With the idea of group presentations, one might ask if there's some way of combining group presentations together. And indeed, that's the core idea of a {\bf Free Product}.

\begin{definition}[Free Product] Let G, H be groups with presentations $G = \langle A_G | R_G \rangle$ and $H = \langle A_H | R_H \rangle$, then we define the free product ($*$) of G and H to be:
\[G * H = \langle A_G \cup A_H | R_G \cup R_H \rangle\]
\end{definition}

\begin{example}
There are several examples of free products that aligns with our pre-existing beliefs.
\begin{enumerate}
    \item {\bf (Free Product of Free Groups)} Let $F(A) = \langle a_1, ..., a_n\ |\ \rangle$, $F(B) = \langle b_1, ..., b_m\ |\ \rangle$, then $F(A)*F(B) = \langle a_1, ..., a_n, b_1, ..., b_n\ |\ \rangle$ is a free group of rank $m+n$.
    \item Let $\mathbb{Z}/n\mathbb{Z} \cong \langle x\ |\ x^n = e\rangle$ and $\mathbb{Z}/m\mathbb{Z} \cong \langle y\ |\ y^m = e\rangle$, then 
    \[\mathbb{Z}/n\mathbb{Z}*\mathbb{Z}/m\mathbb{Z} = \langle x, y\ |\ x^n = e, y^m = e\rangle\]
\end{enumerate}
\end{example}

\begin{definition}[Free Product with Amalgamation] Let G, H, A be arbitrary groups with presentatons $G = \langle A_G | R_G \rangle$, $H = \langle A_H | R_H \rangle$, $A = \langle B | R_A \rangle$ with two injective homomorphisms $f_1: A \to G$ and $f_2: A \to H$, let
\[R = \{f_1(a)f_2(a)^{-1}\ |\ a \in B\}\]
Then we call the free product of G and H with amalgamation A:
\[G *_A H = \{A_G \cup A_H | R_G \cup R_H \cup R\}\]

\noindent The diagram below shows how $A$, $G$, and $H$ relate to each other by the maps $f_1$ and $f_2$ and produce the free product with amalgamation $A$.
% https://q.uiver.app/?q=WzAsNCxbMCwxLCJBIl0sWzEsMCwiRyJdLFsxLDIsIkgiXSxbMiwxLCJHICpfQSBIIl0sWzAsMSwiZl8xIl0sWzAsMiwiZl8yIiwyXSxbMiwzLCIiLDIseyJzdHlsZSI6eyJib2R5Ijp7Im5hbWUiOiJkYXNoZWQifX19XSxbMSwzLCIiLDAseyJzdHlsZSI6eyJib2R5Ijp7Im5hbWUiOiJkYXNoZWQifX19XV0=
\[\begin{tikzcd}
	& G \\
	A && {G *_A H} \\
	& H
	\arrow["{f_1}", from=2-1, to=1-2]
	\arrow["{f_2}"', from=2-1, to=3-2]
	\arrow[dashed, from=3-2, to=2-3]
	\arrow[dashed, from=1-2, to=2-3]
\end{tikzcd}\]

\end{definition}

\begin{example}
Consider the free groups $G = \mathbb{Z}_{4} = \langle a | a^4\rangle$, $H = \mathbb{Z}_{6} = \langle b | b^6 \rangle$, and $A = \mathbb{Z}_{2} = \langle c | c^2 \rangle$.\\\\
To find the free product of $G$ and $H$ with amalgamation $A$, we first ask how we can define injective homomorphisms $f_1: A \to G$ and $f_2: A \to H$. The subgroup of $G$ that is homomorphic to $A$ is $\langle e, a^2\rangle$. The subgroup of $H$ homomorphic to $A$ is $\langle e, b^3 \rangle$. So we define $f_1$ such that $f_1(c) = a^2$ and $f_2$ such that $f_2(c) = b^3$. We want our free product to have $f_1(c) = f_2(c)$ for all $c \in A$, so we let $ R = \{a^2b^{-3}\}$ and obtain the free product of $G$ and $H$ with amalgamation $A$: \[G *_A H = \langle a,b | a^4 = e,b^6=e,a^2b^{-3}=e \rangle\]
\end{example}